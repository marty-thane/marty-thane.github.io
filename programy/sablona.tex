% sablona pro seminarni praci v LuaLaTeX/BibLaTeX
\documentclass[12pt,a4paper,oneside]{article}

% format pro archivaci
\usepackage[a-1b]{pdfx}

% ceska pravidla pro typografii
\usepackage[czech]{babel}
\usepackage[czech=quotes]{csquotes}
\usepackage[nosingleletter]{impnattypo}

% pismo
\usepackage{fontspec}
\setmainfont{TeX Gyre Termes}

% citacni system
\usepackage[
	backend=biber,
	sorting=none,
	style=iso-numeric,
]{biblatex}
% \addbibresource{refs.bib}

% okraje a velikost strany
\usepackage[
	a4paper,
	margin=25mm,
	bindingoffset=5mm,
]{geometry}

% radkovani
\linespread{1.5}
\parskip6pt

% cislovani stran vpravo dole
\usepackage{fancyhdr}
\renewcommand{\headrulewidth}{0pt}
\fancyhead{}
\fancyfoot{}
\fancyfoot[R]{\thepage}

% sekce zacinat na nove strance
\newcommand{\sectionbreak}{\clearpage}

% todo: cislovat az od subsekci

% prikaz pro vkladani obrazku
\usepackage{graphicx}
\newcommand{\putimage}[4][18em]{
	\begin{figure}[h]
		\centering
		\includegraphics[height=#1]{#4}
		\caption{#2}
		\label{#3}
	\end{figure}
}

% zlepseni vzhledu pomoci microtypingu
\usepackage{microtype}

% odkazy
\usepackage{hyperref}

\begin{document}
\pagestyle{empty} % uvodni strany bez cislovani

% titulni strana
\begin{center}
\Large
Univerzita Jana Evangelisty Purkyně
v Ústí nad Labem

Přírodovědecká fakulta

\vspace*{\fill}

\Huge
Název práce

\Large
SEMINÁRNÍ PRÁCE

\vspace{4em} % 2em pokud bez fakulty
\vspace*{\fill}

\end{center}

{\setlength\parindent{0pt} % bez odrazeni
Vypracoval: Petr Novák

Vedoucí práce: Mgr. Lucie Zákravská
\hfill Ústí nad Labem \the\year}

\clearpage

% prohlaseni
\vspace*{\fill}

Tímto prohlašuji, že jsem seminární práci vypracoval samostatně
a s využitím uvedených zdrojů. % nutno predelat

V .................... dne ........................
   
\hspace{8.5cm} \makebox[2in]{\hrulefill}
  				 
\hspace{8.5cm} \makebox[2in]{Podpis autora}	

\clearpage

% abstrakt, klicova slova
{
	\abstract
	\noindent
	Stručné a srozumitelné shrnutí obsahu práce. Abstrakt je nejvýše v rozsahu
	jedné stránky, nesmí obsahovat zkratky, kromě zkratek obecně přijatých. V
	případě závěrečné práce napsané v jazyce anglickém se český abstrakt
	neuvádí.
}

{
	\renewcommand{\abstractname}{Klíčová slova}
	\abstract
	\noindent
	seznam klíčových slov, která nejlépe vystihují obsah práce
}

\clearpage

% obsah
\tableofcontents
\clearpage

% samotny text
\pagestyle{fancy}
% \input{text.tex}
\section{Intro}
\subsection{Ahoj}

\printbibliography

\end{document}
